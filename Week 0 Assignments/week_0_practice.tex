\documentclass[addpoints, 12pt]{exam}
\usepackage[utf8]{inputenc}
\usepackage{amsmath,amsthm,amsfonts,amssymb,amscd}
\usepackage{multirow,booktabs}
\usepackage[table]{xcolor}
\usepackage{fullpage}
\usepackage{lastpage}
\usepackage{enumitem}
% \usepackage{fancyhdr}
\usepackage{mathrsfs}
\usepackage{wrapfig}
\usepackage{setspace}
\usepackage{multicol}
\usepackage[retainorgcmds]{IEEEtrantools}
\usepackage[margin=3cm]{geometry}
\usepackage{listings}
\usepackage{pgf}
\usepackage{pgfplots}
\pgfplotsset{soldot/.style={color=red,only marks,mark=*, mark options={scale=1.5}},
             holdot/.style={color=red,fill=white,only marks,mark=*, mark options={scale=1.5}},
             compat=1.12}
\usepgfplotslibrary{external}
\usepgfplotslibrary{fillbetween}
\usetikzlibrary{intersections, calc, angles, quotes}
\usepackage{floatrow}
\newlength{\tabcont}
\setlength{\parindent}{0.0in}
\setlength{\parskip}{0.05in}
\newcommand\assignment{Week 0 Practice Problems}
\newcommand\school{SouthLake Christian Academy}
\newcommand\course{AP CSP}
\newcommand\examdate{August 2022}

\newcommand*{\texthide}[1]{\underline{\phantom{#1}}}
\newcommand*{\textshow}[1]{\underline{#1}}

\lstdefinestyle{mystyle}{
    basicstyle=\ttfamily\footnotesize,
    breakatwhitespace=false,         
    breaklines=true,                 
    captionpos=b,                    
    keepspaces=true,                 
    numbers=left,                    
    numbersep=5pt,                  
    showspaces=false,                
    showstringspaces=false,
    showtabs=false,                  
    tabsize=2
}
\lstset{style=mystyle}

\pagestyle{head}
\firstpageheader{Name: \\ Section: }{\textbf{\Large \assignment}}{Date: \hspace{5em}}%
\firstpageheadrule

\begin{document}

\begin{center}\textit{Feel free to use the paper tool we made in class!}\end{center}

\begin{questions}

    \question Give an example of one abstraction you use in life, and explain how it's an abstraction.
    Briefly explain the inner workings of this abstraction (use Google!)
    \vfill
		
	\question If you sometimes count using your fingers, odds are you can count up to 5 things on one hand (using 5 fingers).
    But that's if you're using ``unary" notation, whereby you only have a single digit at your disposal, a finger, which you can think of as a \texttt{1}.
    Binary, by contrast, allows you to use two digits, \texttt{0} and \texttt{1}.
    How high could you count on one hand (with 5 fingers) using binary?
    Assume that a raised finger represents a \texttt{1} and a lowered finger represents a \texttt{0}.
	\vfill

	\question Convert the following to decimal. Remember that ``\texttt{0b}'' just indicates that the number is in binary!
    \textbf{You must show all your work!}
    \begin{parts}
        \part \texttt{0b1001}
        \vfill
        \part \texttt{0b1110}
        \vfill
        \part \texttt{0b0011}
        \vfill
    \end{parts}
	

    \question If a computer only had 4-bit hardware, what is the result of \texttt{0b1101 + 0b0011}?
    Indicate whether an overflow, underflow, or no error occured.
    \vfill

\end{questions}

\end{document}